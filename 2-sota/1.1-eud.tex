\section{End-user Development}
\subsection{What is EUD?}
- End-User Development can be defined as a set of methods, techniques, and tools that allow users of software systems, who are acting as non-professional software developers, at some point to create, modify or extend a software artefact. \\
- main goal of EUD: empowering end-users to develop and adapt systems themselves\\
- must be made considerably more flexible and they must support the demanding task of EUD: they must be easy to understand, to learn, to use, and to teach\\
- easy to test and assess their EUD activities\\
 (\cite{lieberman2006end})
 
\subsection{Types of EUD}
1. Parameterisation or Customisation 

2. Program Creation and Modification : Programming by Example, Incremental Programming, Model-based development, Extended annotation or parameterisation .\\
 
\subsection{EUD interaction styles} 

Programming is.....\\
- using visual attributes\\
- by demonstration\\
- by specification\\
- with text\\
 (Scaffidi2011enduser)\\
 
 \subsection{Design guidelines for EUD systems}
  (\cite{ko2004six}), (\cite{repenning2006makes})
 
\subsection{Definitions}
- Programming\\
- Interactive vs Batch\\
- Visual programming (VP): graphics are the program itself, e.g. flow charts and graphical programming languages, Scratch, \\
- Program visualization (PV): program is specified in textual manner and graphics illustrate some aspect of the program, \\
- Programming by Example\\
 (Myers1986visual) defines VP vs PbE vs batch/interactive
 
\subsection{Challenges of EUD}
- limited human patience and inconsistent user input
 
 
\subsection{Design choices to implement interactive machine learning methods}
(\cite{chernova2014robot})
Where to have teacher's input? Where to have robot influence/decide?\\
- Data collection: what training data to include? \\
- Selecting the feature space and its structure: include input features that are in fact discriminatory \\
- Defining a reward function that represents the task learned \\
- Subtasking the problem: determine the task structure \\

