\subsection{Knowledge Engineering Tools}\label{subsec:Knowledge Engineering}
Defining domain models from scratch can be expensive and error-prone, even for domain experts.
Existing knowledge engineering (KE) tools for AI planning can automatically encode a domain model from observed plans.
\cite{jilani2014automated} evaluates the quality and efficiency of nine state-of-the-art tools by comparing input requirements, provided output, user experience, availability, and other criteria.
GIPO (\cite{simpson2007planning}), a knowledge engineering tool which simplifies the creation of planning domain models using a graphical interface, has been identified as the only beginner-friendly system.
GIPO uses Opmaker2 (\cite{mccluskey2009automated}), a knowledge acquisition and formulation tool which generates a set of PDDL action schema from a given partial domain model and training sequence.
LOCM (Learning Object Centred Models) (\cite{cresswell2013acquiring}) is a system which can learn action schema from planning traces only.
itSimple (\cite{vaquero2013itsimple}) was first developed in 2004 and is a KE tool which takes a input in the form of UML (Unified Modeling Language) (\cite{omg2005unified}) and generates representations in Petri Nets (\cite{murata1989petri}) and PDDL. The authors have been consistently working on improvements to allow new features, with the latest version published in 2013.