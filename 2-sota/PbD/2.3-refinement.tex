\subsection{Interactive Policy Refinement Techniques}\label{subsec:Other RP Methods}
As human demonstrations are often noisy or suboptimal, interactive policy refinement techniques can be used.
Learning by exploration allows the robot to refine the policy on its own and can be combined with  teacher critique. 

The robot can learn from observation data provided by the teacher (Learning by Observation), by self-exploration using a defined reward function (Learning by Exploration), or by interactive teaching from continuous feedback (Active Learning). In the following sections we will give an overview of these approaches.

In \cite{nicolescu2003natural} the teacher can refine tasks by inserting or removing behaviours from the networks of abstract behaviours.

\subsubsection{Learning by Exploration}\label{sssec:LbExploration}

- the robot acquires data from interaction with its environment
- need a reward function which allows him to learn from the data
1. Reinforcement learning (\cite{sutton1998reinforcement}, \cite{mnih2015human})
 - reward function can be specified by the user, robot learns policy by self-exploration
 2. Inverse reinforcement learning (\cite{abbeel2004apprenticeship})
 - robot is given teacher demonstrations and learns reward function

\subsubsection{Active/Interactive Learning}\label{sssec:Active Learning} 
For active or interactive learning, the teacher provides robot initial data, observes robot performance and provides feedback (\cite{chernova2014robot},\cite{calinon2007active}).
The robot can ask for feedback (\cite{cakmak2012aaai}) and the teacher can modify the learned action using a visual interface (\cite{alexandrova2015roboflow}).
(\cite{nicolescu2003natural}) present a PbD approach, which allows the robot to learn skill representations and refine them by using feedback cues provided by the teacher.
Similarly, (\cite{calinon2007active}) and (\cite{calinon2007incremental}) implement systems, which actively involve the teacher in the robot's learning process, by providing human guidance to a humanoid robot. 
The robot first observes the demonstration performed by the teacher, who is wearing motion sensors. When it tries to reproduce the action, the teacher can refine the movement by physically moving its limbs.
\cite{martinez2017relational} propose a relational RL solution with guided demonstrations where the robot requests for help from the human teacher to reduce the learning time.
However, demonstrations are only requested if they yield significant improvements as the teacher's time is considered more valuable than the robot's time.

%\subsubsection{Robot feedback}
%Robot feedback is important to provide the teacher with enough information about what to demonstrate to the robot.
%Reasons such as wrong preconditions or missing action effects can lead to failures.
%Failures can be explained using excuses \cite{?}
%Benchmark: \cite{martinez2017relational} uses 3 examples from Planning competitions


