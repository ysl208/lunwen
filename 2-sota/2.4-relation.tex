\subsection{Relation to present work}

Existing PbD implementations are rather task-specific and cannot be applied to arbitrary scenarios.
In general, many robot PbD implementations already have pick and place actions coded into the system \cite{veeraraghavan2008teaching}.
Often, the robot has a single goal specified, which it tries to achieve given a pre-defined, possibly iterative, action sequence.
When the process of the task changes, e.g. to include a new subtask or to achieve a different goal, the robot needs to be reprogrammed.
This highlights a key issue in PbD to design a generic system that allows the learning of tasks applicable to different scenarios.
Unlike other PbD implementations, our system does not have any atomic actions preprogrammed but allows the user to construct any action model.
Furthermore, current implementations include an intermediate manual step to construct action sequences in order to achieve a goal.

We want to equip it with all the skills needed to act autonomously in any state of the world, rather than teaching it how to react to a given state of the world.
To achieve this, we have to design our framework using plans, which provides us with the required tools.
As we focus on collaborative environments, where the teacher is actively involved in the learning process, plans allow a means to communicate the teacher's intentions together with the demonstrations.
Therefore, our policy derivation choice will enable the robot to learn the action, as well as its semantics, and combine them with goal-oriented problem-solving techniques.
