The increasing presence of robots in industries has not gone unnoticed. 
Cobots (collaborative robots) are revolutionising industries by allowing robots to work in close collaboration with humans.
Large industrial players have incorporated them into their production lines, but smaller companies hesitate due to high initial costs and the lack of programming expertise. 
In this thesis we introduce a framework that combines two disciplines, Programming by Demonstration and Automated Planning, to allow users without programming knowledge to program a robot. 
The user constructs the robot's knowledge base by teaching it new actions by demonstration, and associates their semantic meaning to enable the robot to reason about them. 
The robot adopts a goal-oriented behaviour by using automated planning techniques, where users teach action models expressed in a symbolic planning language (PDDL).

In this thesis we present preliminary work on user experiments using a Baxter Research Robot to evaluate our approach.
We conducted qualitative user experiments to evaluate the user's understanding of the symbolic planning language and the usability of the framework's programming process.
We showed that users with little to no programming experience can adopt the symbolic planning language, and use the framework.

We further present our work on a Programming by Demonstration system used for organisation tasks.
The system includes a goal inference model to accelerate the programming process by predicting the user's intended product configuration.

\paragraph{Keywords:} Robotics, Cobotics, End-User Programming, Programming by Demonstration, Automated Planning