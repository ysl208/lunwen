La robotique est de plus en plus présente dans l’industrie. Les cobots (robots collaboratifs), travaillant en collaboration avec l’homme, sont en train de révolutionner l’industrie. Les grands acteurs industriels les ont déjà intégrés dans leurs chaînes de production, mais les petites entreprises hésitent encore en raison des coûts élevés des investissements nécessaires et de leur manque d'expertise dans leur maintenance et leur programmation. Des travaux récents se sont concentrés sur la possibilité, pour des utilisateurs finaux, de programmer des cobots, mais la programmation des actions qu'ils peuvent réaliser reste encore un problème d’expert en robotique.
Dans cette thèse, nous proposons un système qui combine deux disciplines de l'IA, l'apprentissage par démonstration et la planification automatique, pour permettre à des utilisateurs ayant peu ou pas de connaissances en programmation. Dans un premier temps, l'utilisateur apprend de nouvelles actions au cobot par démonstration et leur associe une sémantique via une interface graphique. Dans un second temps, le cobot est capable en utilisant des algorithmes de planification automatique de raisonner sur la sémantique des actions apprise de calculer dynamiquement  la séquence d’actions à réaliser pour atteindre un objectif qui lui a été confié.
Nous présentons dans ce manuscrit tout d'abord des travaux préliminaires pour évaluer la faisabilité de notre approche. Ces travaux préliminaires s’appuient sur une expérimentation qualitative réalisée auprès d'utilisateurs afin d'évaluer leur compréhension du langage de planification symbolique utilisé pour définir la sémantique des actions apprises au cobot. Nous avons montré que les utilisateurs ayant peu ou pas d'expérience en programmation peuvent adopter ce langage et comprendre le processus de programmation proposé.
Nous présentons ensuite une méthode pour apprendre au cobot par démonstration un objectif à réaliser. Nous avons évalué notre approche sur une tâche de packaging classique en robotique.
Finalement, sur la base des résultats obtenus, nous présentons une implémentation de notre système de programmation par démonstration, appelé iRoPro, un système fonctionnel complet qui permet d'apprendre à un cobot des actions de bas niveau (trajectoire) et de haut niveau (sémantique) par démonstration capable d’être manipulée par un planificateur de tâches.  Notre système a été évalué  et validé par une étude d’utilisateurs. Nous avons montré que les utilisateurs avec des connaissances en programmation limitées pouvaient relativement facilement apprendre et utiliser notre système de programmation par démonstration.