La présence croissante des robots dans les industries n'est pas passée inaperçue. 
Les cobots (robots collaboratifs) révolutionnent les industries en permettant aux robots de travailler en collaboration avec l'homme.
Les grands acteurs industriels les ont intégrés dans leurs chaînes de production, mais les petites entreprises hésitent en raison des coûts initiaux élevés et surtout du manque d'expertise en programmation. 
Les travaux récents se sont concentrés sur la possibilité pour les utilisateurs finaux de programmer des robots, mais apprendre des actions réutilisables est généralement laissé à des experts en robotique.
Dans cette thèse, nous proposons un système qui combine deux disciplines, la programmation par démonstration et la planification automatique, pour permettre à des utilisateurs ayant peu ou pas de connaissances techniques de programmer un robot. 
L'utilisateur construit la base de connaissances du robot en lui apprenant de nouvelles actions par démonstration, et associe leur signification sémantique via une interface graphique pour permettre au robot de raisonner à leur sujet. 
Le robot adopte un comportement orienté objectif en utilisant des techniques de planification automatisées pour réutiliser les actions apprises afin de générer des solutions pour des tâches inédites.
Nous présentons d'abord des travaux préliminaires en termes d'expériences d'utilisation à l'aide d'un robot de recherche Baxter pour évaluer la faisabilité de notre approche.
Nous avons mené des expériences qualitatives auprès des utilisateurs afin d'évaluer leur compréhension du langage de planification symbolique et de la convivialité du processus de programmation du système proposé.
Nous avons montré que les utilisateurs ayant peu ou pas d'expérience en programmation peuvent adopter le langage de planification symbolique et comprendre le processus de programmation proposé.
Nous présentons ensuite un système de programmation orienté objectif pour les tâches d'organisation robotique d'étagères qui utilise la programmation par démonstration pour enseigner simultanément les objectifs et les actions.
Sur la base des résultats obtenus et du système développé, nous présentons une implémentation d'iRoPro, un système fonctionnel de bout en bout qui permet d'enseigner des actions de bas et haut niveau par démonstration à réutiliser avec un planificateur de tâches.
Nous évaluons le pouvoir de généralisation du système et montrons comment les actions enseignées peuvent être réutilisées pour des tâches plus complexes.
Enfin, nous avons validé la convivialité du système par une étude d'utilisateurs et démontré que les utilisateurs de tous niveaux de programmation et de formation peuvent facilement apprendre et utiliser notre système de programmation de robot.

%\textbf{Mots-clés:} Robotique, Cobotique, Programmation utilisateur final, Programmation par démonstration, Planification automatisée

%Mots-clés : Robotique, Cobotique, Programmation utilisateur final, Programmation par démonstration, Planification automatisée 