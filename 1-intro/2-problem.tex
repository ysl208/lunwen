\section{Summary of contributions}
%\subsection{Thesis statement}
%Non-expert users can program a robot to complete a set of tasks by teaching it atomic actions by demonstration and correctly assigning them preconditions and effects that can be used with an automated planner to solve more complex problems.

In this thesis we propose a robot programming framework that allows human operators to:
\begin{itemize}
	\item teach the robot action models in a comprehensive automated planning representation, translated into PDDL, and
	\item enable it to use the learned action models to be controlled with a goal-oriented approach based on automated planning techniques.
\end{itemize}

With the focus on cobotic environments, we want to discover the difficulties encountered by the user, when faced with techniques in both Robot Programming by Demonstration and Automated Planning domains. 
The integration of a functioning real-world system would require solutions in perception (e.g. object identification), motion planning (e.g. manipulation, navigation, safety), human–robot interaction, and cognitive robotics (e.g. action learning, task planning). 
The integration of all aspects into a real-time system is a major challenge.
In this thesis we address this challenge and provide partial solutions for the other areas.
The contributions of this thesis can be summarised as follows:
\begin{itemize}
	\item {A robot programming framework that combines PbD and Automated Planning, where the robot learns action models by demonstration, and the problem of finding an action sequence is delegated to a planner.
	The robot programming process consists of steps:
	\begin{enumerate}
		\item the non-expert user demonstrates atomic actions to the robot, and teaches \textit{action models}, expressed in a symbolic planning language (STRIPS \cite{fikes1971strips}),
		\item the robot uses these action models with an automated planner to generate solutions to user-defined goals,
		\item the user can revisit the taught action model to refine them.
	\end{enumerate}}
    \item {Experimental findings on user acceptance of the proposed robot programming framework, when users are tasked to teach a robot by demonstration and correctly assign action conditions used for automated planning. \\
    Research question: \textit{Can users teach a robot action models for automated planning using the robot programming framework?}}
    \item {Experimental findings on issues encountered when non-experts are tasked to use AI planning and PDDL concepts to describe a domain model to a robot. 
    	We evaluated the user's ability to construct symbolic action models, in terms of preconditions and effects, used by automated planners.\\ 
    Research question: \textit{How do non-expert users adopt the automated planning language with its action model representation?}}
    \item {System for robotic shelf organisation tasks: simultaneous end-user programming of goals and actions using PbD and interactive visualisation. \todo{elaborate}}
\end{itemize}
