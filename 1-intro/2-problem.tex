\section{Summary of contributions}
%\subsection{Thesis statement}
%Non-expert users can program a robot to complete a set of tasks by teaching it atomic actions by demonstration and correctly assigning them preconditions and effects that can be used with an automated planner to solve more complex problems.
%\todo{2-3 key research ideas/questions}
%In this thesis we propose a robot programming framework that allows human operators to:
%\begin{itemize}
%	\item teach the robot action models in a comprehensive automated planning representation, translated into PDDL, and
%	\item enable it to use the learned action models to be controlled with a goal-oriented approach based on automated planning techniques.
%\end{itemize}


The contributions of this thesis can be summarised as follows:
\begin{itemize}
	\item {A proposed robot programming framework that combines PbD and Automated Planning, where the robot learns action models by demonstration, and the problem of finding an action sequence is delegated to a planner (\cite{liang2017framework}).
	The robot programming process consists of the following steps:
	\begin{enumerate}
		\item the non-expert user teaches the robot atomic actions by kinesthetic demonstration, and defines {action models} by associating preconditions and effects,
		\item the robot uses these action models with an automated planner to generate solutions to user-defined goals,
		\item the user can revisit the taught action models to refine them.
	\end{enumerate}
	\textbf{Research question:} \textit{Can end-users teach the robot actions and associated conditions using PbD that can be used by a planner to solve a user-defined task?}}
    \item {Experimental findings on user acceptance of the proposed framework. Users were tasked to teach a robot by kinesthetic demonstration and correctly assign action conditions that can be used for automated planning.\\
    \textbf{Research question:} \textit{Can users teach a robot action models for automated planning using the robot programming framework?}}
    \item {Experimental findings on issues encountered when non-experts are tasked to use AI planning concepts and a symbolic planning language to describe the world state to a robot. 
    We evaluated the user's ability to construct symbolic action models, in terms of preconditions and effects, used by automated planners (\cite{liang2017evaluation}).\\
    \textbf{Research question:} \textit{How do non-expert users adopt the automated planning language with its action model representation?}}
    \item {A system for robotic shelf organisation tasks: simultaneous end-user programming of goals and actions using PbD and interactive visualisation (\cite{liang2018simultaneous}).
    The system allows the robot to learn and execute organisation tasks from user input via kinesthetic demonstration and a graphical interface.
    We teach the robot \textit{what} and \textit{how} to perform a task using PbD and a goal inference model.
    We evaluated user teaching strategies with experiments on Amazon Mechanical Turk and compared their performance to eight benchmark strategies.\\
    \textbf{Research question:} \textit{Can users teach the robot an organisation task using PbD and goal inference and what teaching strategies do users prefer?}}
\end{itemize}
