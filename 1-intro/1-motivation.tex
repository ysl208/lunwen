\section{Context}
% 1-2 pages max
Technologies have revolutionised the manufacturing industry ever since the Industrial revolution in the 19th century.
Robots are increasing productivity by replacing humans for arduous and repetitive manual tasks.
%are increasing productivity, efficiency, lower costs, and relieve humans from operationally dangerous tasks.
Increasing order requests for industrial robots has led to higher capital investments into the field.
Although robots can be superior in automating human tasks, there remain many tasks that cannot be completely taken over, such as high-precision tasks, and still need human intervention.
To allow both human precision and robot automation, collaborative robots have been introduced.

\subsection{Cobotics}\label{subsec:Cobotics}
Collaborative robots, or \textit{cobots}, have been introduced by \cite{colgate1999cobots} and allow for a close collaboration between humans and robots.
They enable humans to perform tasks, which they cannot perform on their own, due to physical constraints such as the manipulation of heavy parts.
Furthermore, they reduce risks of work-related accidents, including health hazards such as exposure to dangerous environments (\eg chemical acids, excessive temperatures or noise), as well as sleeping disorders caused by rotating work shifts.
Cobots contribute to productivity gains as they are designed to respond to actions of the human operator and take over manual and repetitive tasks.
While replacing jobs of low-skilled human workers, they open up a market for new high-skilled jobs.

Cobotic systems have been adopted in several industries from the food-processing industry (\cite{Food}), to aeronautics (\cite{Airbus}), to the health industry (\cite{Ebola}).
However, companies resist the use of robots in their daily routines as they consider the investment cost ineffective, due to their high initial costs and the lack of trained personnel.
%, who have the needed programming skills to fully operate and exploit the robots.
Traditional robot programming solutions require domain experts and robots are generally programmed to complete a specific task.
Recent approaches using large amounts of data (\eg Neural Networks (\cite{billard2001robust})) or self-exploration (\eg Reinforcement Learning (\cite{smart2002effective})) become infeasible for task-specific applications.
This is a bottleneck for industries and many robot programming solutions fail as the deployment in real world scenarios introduces further limitations. 
Thus, recent research has been focusing on robot programming for end-users, in particular Robot Programming by Demonstration.

\subsection{Robot Programming by Demonstration}
Robot Programming by Demonstration (PbD), also referred to as \textit{Learning from Demonstration}, is an end-user programming technique for teaching a robot new skills by demonstrating a task, without writing code (\cite{billard2008robot}).
Influenced by natural learning paradigms in humans and other animals, it is an intuitive robot programming method, with the goal to refine the robot's performance, by providing repetitive demonstrations.
PbD has become a central topic in research areas, with the aim to move from purely pre-programmed robots to flexible user-based interfaces for training robots.

Figure \ref{fig:Principle Overview} shows the life-cycle for teaching a robot by demonstration.
The teacher demonstrates the desired behaviour to the robot.
The robot uses its sensors for a multi-modal perception of the demonstration and extracts the relevant information to create a model of the skill.
The new skill is executed in a new context and evaluated by the teacher.
The teacher can refine the learned skill by providing additional demonstrations of the same skill.
The robot then generalises over the demonstrations by extracting relevant features that remained unchanged across the set of demonstrations.
This incremental learning process allows the robot to acquire a new skill from demonstrations provided by non-roboticists.

\begin{figure}[h]
	\centering
	\includegraphics[width=0.63\linewidth]{figures/PbD-Overview}
	\caption{PbD Overview}
	\label{fig:Principle Overview}
\end{figure}

%Besides the advantage of being able to teach the robot tasks without the need to write code, PbD provides a powerful tool to improve learning abilities by reducing the search space of possible solutions.
However, learning object manipulation tasks is still considered a hard problem, as the robot has limited knowledge about the world and restricted sensor availability (\cite{ekvall2008robot}).
Many PbD algorithms have been proposed in the literature (\cite{argall2009survey,billing2010formalism}), but there still remain several challenges such as the suboptimality of demonstrations (\cite{chen2003programing,kaiser1995obtaining}) or the lack of comparative user studies (\cite{suay2012practical}).

% teaching full action sequences
Another major problem is that the robot is generally demonstrated an action sequence to complete a specific task (\cite{orendt2016robot,peppoloni2014ros}).
Take for example the Tower of Hanoi problem (\cite{douglas1985metamagical}) as shown in \fig{fig:Tower of Hanoi}a.
The objective is to move the entire stack from one rod to another, obeying the following rules:
\begin{itemize}
\item Only one disk can be moved at a time.
\item Each move consists of taking the upper disk from one of the stacks and placing it on top of another (possibly empty) stack.
\item No disk may be placed on top of a smaller disk.
\end{itemize}

The robot can be taught an action sequence to solve the problem for three disks.
When the problem changes to four disks (\fig{fig:Tower of Hanoi}b), the robot has to be demonstrated a new sequence, even though both problems obey the same rules.
This reprogramming process can be complicated and time-consuming as it does not allow to generalise for different tasks.
A more efficient approach would be to teach the robot the primitive action of moving a disk, associate rules or \textit{conditions} to this action (\eg smaller disks can only be placed on top of larger ones), and have the robot generate an optimal solution using an automated planner \cite{ghallab2004automated}.

\begin{figure}[htp]
	\centering
	\includegraphics[width=.4\textwidth]{figures/hanoi-0}\hspace{2cm}%\hfill
	\includegraphics[width=.4\textwidth]{figures/hanoi-1}
	\caption{Tower of Hanoi problem a) with three disks b) with four disks}
	\label{fig:Tower of Hanoi}
\end{figure}

\subsection{Automated Planning}
Automated Planning, also known as \textit{AI Planning}, is a research field that focuses on the development of efficient search algorithms to generate solutions to problems.
Given a set of actions, a description of the state of the world, and some goal state, the planner generates a sequence of actions, which guarantees the transition from the initial state to the goal state (\fig{fig:Planning domain and action}).
To allow a correct transition between different states of the world, actions are defined in terms of preconditions and effects (\fig{fig:Planning domain and action}b). 
Planning algorithms use a symbolic planning language such as STRIPS (\cite{fikes1971strips}) or PDDL (\cite{ghallab2004automated}) as their standard encoding language.
Thus, the Tower of Hanoi problem could be defined as a planning problem and solved for any number of disks using an automated planner.
%In the initial state, the disks are stacked in ascending order (smallest at the top) on one of the pegs, in the goal state the disks are stacked in the same order on one of the other two remaining pegs. 
%The goal state must be achieved by obeying certain rules for moving the disks (\eg only the top disk of a stack can be moved at a time and may not be placed on top of a smaller disk). 


\begin{figure}[htp]
	\centering
	\includegraphics[width=.55\linewidth]{figures/PbD-AutomatedPlanner}
	\includegraphics[width=.4\linewidth]{figures/schema-logic-1}
	\caption{Planning domain describing an initial world state, goal state, and a generated action sequence (left), where actions are described in terms of preconditions and effects (right).}
	\label{fig:Planning domain and action}
\end{figure}
