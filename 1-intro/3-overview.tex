\section{Thesis Overview}
\subsection{Thesis statement}
Non-expert users can program a robot to complete a set of tasks by teaching it atomic actions by demonstration and correctly assigning them preconditions and effects that can be used with an AI planner to solve more complex problems.

\subsection{Summary of contributions}
The contributions of this thesis can be summarised as follows:
\begin{itemize}
    \item Experimental findings on user acceptance of a robot programming framework when users are tasked to teach a robot by demonstration and correctly assign action conditions used for AI planning.
    \item Experimental findings on issues encountered when non-experts are tasked to use AI planning and PDDL concepts to describe a domain model to a robot.
    \item System for robotic shelf organisation tasks: simultaneous end-user programming of goals and actions using PbD and interactive visualisation.
\end{itemize}

\subsection{Document outline}
This thesis is organised as follows. 

\chapt{chap:Sota} presents the state of the art in the domains End-user Robot Programming, Programming by Demonstration, and Automated Planning.

\chapt{chap:Contribution} presents our proposed solution as a Framework for Robot Programming for End-Users.

\chapt{chap:Implementation} details our achieved implementation of the framework. 

\chapt{chap:Evaluation} evaluates our framework in terms of experiments and presents their results. We first present experimental findings on user acceptance and issues encountered when introduced to the AI planning concepts.

\chapt{chap:Conclusion} concludes our work and discusses possibilities for future implementations of the framework.
