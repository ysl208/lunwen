\section{Document organization}
This thesis is organised into two parts:
%An overview of the chapters can be seen in Figure \ref{fig:chapter-overview}.

The first part (\ref{part1}) provides a literature review of state of the art techniques in robot programming (\textbf{\chapt{chap:Sota}}), followed by an overview of Automated Planning (\textbf{\chapt{chap:Sota-AP}}).

The second part (\ref{part2}) presents our contributions involving iRoPro, the proposed end-user robot programming framework. %\textbf{\chapt{chap:Contribution}}. 
We first conduct qualitative experiments and present experimental findings on user acceptance and issues encountered when introduced to the AI planning concepts (\textbf{\chapt{chap:Pre-Experiments}}).
We then focus on evaluating our goal-oriented end-user robot programming approach (\textbf{\chapt{chap:OrganisingTasks}}) and present our work on shelf organising tasks to simultaneous program goals and actions by demonstration.
Our main contribution is an end-to-end system implementation of iRoPro, for which we present details on the implementation and the system evaluation in \textbf{\chapt{chap:Implementation}}.
Finally, we conclude this thesis in \textbf{\chapt{chap:Conclusion}} and discuss possibilities for future implementations of the framework.

%
\tikzstyle{format} = [draw, thin, fill=blue!20]
\tikzstyle{medium} = [ellipse, draw, thin, fill=green!20, minimum height=2.5em]

\begin{figure}
	\begin{tikzpicture}[node distance=3cm, auto,>=latex', thick]
	% We need to set at bounding box first. Otherwise the diagram
	% will change position for each frame.
	\path[use as bounding box] (-1,0) rectangle (10,-2);
	\path[->]<1-> node[format] (tex) {.tex file};
	\path[->]<2-> node[format, right of=tex] (dvi) {.dvi file}
	(tex) edge node {\TeX} (dvi);
	\path[->]<3-> node[format, right of=dvi] (ps) {.ps file}
	node[medium, below of=dvi] (screen) {screen}
	(dvi) edge node {dvips} (ps)
	edge node[swap] {xdvi} (screen);
	\path[->]<4-> node[format, right of=ps] (pdf) {.pdf file}
	node[medium, below of=ps] (print) {printer}
	(ps) edge node {ps2pdf} (pdf)
	edge node[swap] {gs} (screen)
	edge (print);
	\path[->]<5-> (pdf) edge (screen)
	edge (print);
	\path[->, draw]<6-> (tex) -- +(0,1) -| node[near start] {pdf\TeX} (pdf);
	\end{tikzpicture}
\end{figure}


%\begin{figure}[h]
%	\includegraphics[width=\linewidth]{figures/chapter-overview.png}
%	\caption{Chapter overview of the thesis}
%	\label{fig:chapter-overview}
%\end{figure}
