\section{Contributions}
In this project we proposed a framework that uses programming by demonstration techniques to teach a robot a goal-oriented problem-solving behaviour. Our framework combines the use of two disciplines, Programming by Demonstration and Automated Planning, to create a means for ordinary users to program a robot. Unlike existing PbD approaches, where a task execution is taught to fulfill a goal, our approach will provide the robot with all skills needed to generate the task execution itself. The user teaches the robot not only the action, but also its underlying meaning in terms of preconditions and effects. This permits the robot to understand when an action can be executed, as well as its effects on the state of the world. 

Due to the limited scope of this project, we focused on evaluating the usability of our framework, rather than implementing all proposed functionalities. Thus, we only implemented the main components of the framework, using a Baxter Research Robot, that allow us to conduct realistic experiments. At the end of our implementation phase, Baxter was able to search objects on the table by their colour, learn a new action with user-defined predicates, and translate them into PDDL format. Furthermore, Baxter could reproduce the learned actions in a new context and execute a sequence of actions from an externally defined source.

As we target cobotic environments, we evaluated the usability in terms of experiments simulating an assembly line in a manufacturing environment. Using the wizard-of-oz approach, we managed to create an experimental scenario, which allowed us to obtain realistic feedback from participants. We wanted to evaluate user's adaptation of the learning by demonstration process and their ability to comprehend logical representations used in automated planning techniques. Our experiments showed that users with and without programming experience understood the concept of learning by demonstration, as well as the notion of predicates (preconditions and effects), despite learning about them for the first time. As expected, none of the users managed to deduce immediately all predicates needed for the definition of a pick-and-place action model. Users needed to be faced with different action reproduction scenarios in order to deduce all relevant predicates. A future implementation of this framework would benefit from a user interface, which presents simple examples to provide guidance for new users.
Overall, users were satisfied with the learning by demonstration process and the robot's ability to comprehend the taught predicates.

In conclusion, we believe that our framework offers a solution for users without programming experience to efficiently program a robot, by combining the techniques of the two main disciplines: Programming by Demonstration and Automated Planning. Rather than teaching specific task executions, we take a goal-oriented problem-solving approach, which has not been adopted in any existing PbD implementations. As shown in our evaluation phase, the robot's learning process and logical representations are intuitive enough for unexperienced users to understand. Thus, we believe that it is worth pursuing a complete implementation of our framework in the future.

