\section{Contributions}
In this thesis we proposed a framework that uses PbD to teach a robot a goal-oriented problem-solving behaviour.
The framework combines solutions from PbD and Automated Planning, allowing non-expert users to teach a robot new actions together with their semantic meaning.
The user teaches the robot not only the action, but also its underlying meaning in terms of preconditions and effects.
This allows the robot to understand when an action can be executed and therefore use symbolic planners to generate solutions autonomously.
Unlike existing PbD approaches, where a task execution is taught to reach a goal, our approach provides the robot with all skills needed to generate the task execution itself.

As a first step we focused on evaluating the usability of our framework, rather than implementing all proposed functionalities.
Thus, we implemented the main components of the framework, using a Baxter Research Robot, that allowed us to conduct initial experiments.
%At the end of our implementation phase, Baxter was able to search objects on the table by their colour, learn a new action with user-defined predicates, and translate them into PDDL format.
%Furthermore, Baxter could reproduce the learned actions in a new context and execute a sequence of actions from an externally defined source.
%As we target cobotic environments, we evaluated the usability in terms of experiments simulating an assembly line in a manufacturing environment.
%Using the wizard-of-oz approach, we managed to create an experimental scenario, which allowed us to obtain realistic feedback from participants.
We evaluated the user's adoption of the robot programming process and their ability to comprehend symbolic representations used in automated planning techniques.
Our experiments showed that users with and without programming experience understood the concepts of PbD and automated planning, despite learning about them for the first time.
%As expected, none of the users managed to deduce immediately all predicates needed for the definition of a pick-and-place action model.
%Users needed to be faced with different action reproduction scenarios in order to deduce all relevant predicates.
%A future implementation of this framework would benefit from a user interface, which presents simple examples to provide guidance for new users.
%Overall, users were satisfied with the robot programming process and the robot's ability to comprehend the taught predicates.

We further presented work on a system for teaching robots organisation tasks using PbD and goal inference. 
We evaluated this system in terms of user experiments on Amazon Mechanical Turk and compared users' teaching strategies.
This work focused on shelf organisation tasks and demonstrated the representational power of a PbD system for end-users. 
The same PbD system will be used for our proposed robot programming framework.

%In conclusion, we believe that our framework offers a solution for users without programming experience to efficiently program a robot, by combining the techniques of the two main disciplines: Programming by Demonstration and Automated Planning.
%Rather than teaching specific task executions, we take a goal-oriented problem-solving approach, which has been neglected in existing PbD implementations.
%As shown in our evaluation phase, the robot's learning process and logical representations are intuitive enough for non-expert users to understand.

\section{Conclusion} 
\label{sec:conclusion}
In this work we presented iRoPro, an interactive Robot Programming system that allows simultaneous teaching of low- and high-level actions by demonstration.
The robot reuses the actions with a task planner to generate solutions to complex tasks that go beyond the demonstrated action.
The approach was implemented on a Baxter robot and we showed its generalisability on six benchmark tasks by teaching a minimal set of primitive actions that were reused for all tasks.
We further demonstrated its usability with a user study (N=21) where participants with diverse educational backgrounds and programming levels learned how to use the system in less than an hour.
Both user performance and feedback confirmed iRoPro's usability, with the majority ranking it as `acceptable' and half being promoters.
Overall, we demonstrated that our approach allows users with any programming level to efficiently teach robots new actions that can be reused for complex tasks.
Future work will focus on exploring more challenging domains by including a wider range of predicates and probabilistic techniques to improve the condition inference.