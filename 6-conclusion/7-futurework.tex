\section{Contributions}
\label{sec:contributions}
In this thesis we proposed a framework that uses PbD to teach a robot a goal-oriented problem-solving behaviour.
The framework combines solutions from PbD and Automated Planning, allowing non-expert users to teach a robot new actions together with their semantic meaning.
The user teaches the robot not only the action, but also its underlying meaning in terms of preconditions and effects.
This allows the robot to understand when an action can be executed and therefore use symbolic planners to generate solutions autonomously.
Unlike existing PbD approaches, where a task execution is taught to reach a goal, our approach provides the robot with all skills needed to generate the task execution itself.

As a first step we focused on evaluating the usability of our framework, rather than implementing all proposed functionalities.
Thus, we implemented the main components of the framework, using a Baxter Research Robot, that allowed us to conduct initial experiments.
%At the end of our implementation phase, Baxter was able to search objects on the table by their colour, learn a new action with user-defined predicates, and translate them into PDDL format.
%Furthermore, Baxter could reproduce the learned actions in a new context and execute a sequence of actions from an externally defined source.
%As we target cobotic environments, we evaluated the usability in terms of experiments simulating an assembly line in a manufacturing environment.
%Using the wizard-of-oz approach, we managed to create an experimental scenario, which allowed us to obtain realistic feedback from participants.
We evaluated the user's adoption of the robot programming process and their ability to comprehend symbolic representations used in automated planning techniques.
Our experiments showed that users with and without programming experience understood the concepts of PbD and automated planning, despite learning about them for the first time.
%As expected, none of the users managed to deduce immediately all predicates needed for the definition of a pick-and-place action model.
%Users needed to be faced with different action reproduction scenarios in order to deduce all relevant predicates.
%A future implementation of this framework would benefit from a user interface, which presents simple examples to provide guidance for new users.
%Overall, users were satisfied with the robot programming process and the robot's ability to comprehend the taught predicates.

We further presented work on a system for teaching robots organisation tasks using PbD and goal inference. 
We evaluated this system in terms of user experiments on Amazon Mechanical Turk and compared users' teaching strategies.
This work focused on shelf organisation tasks and demonstrated the representational power of a PbD system for end-users. 
The same PbD system will be used for our proposed robot programming framework.

%In conclusion, we believe that our framework offers a solution for users without programming experience to efficiently program a robot, by combining the techniques of the two main disciplines: Programming by Demonstration and Automated Planning.
%Rather than teaching specific task executions, we take a goal-oriented problem-solving approach, which has been neglected in existing PbD implementations.
%As shown in our evaluation phase, the robot's learning process and logical representations are intuitive enough for non-expert users to understand.


\section{Discussions and Open Questions}
\label{sec:discussions}

\subsection{Challenges in PbD}\label{sssec:Challenges in PbD}
Despite ongoing research in PbD, there are several challenges that still need to be addressed:

\textbf{Suboptimality of demonstrations.}
A known PbD problem exists with regards to the type and quality of the demonstration which is dependent on the teacher's knowledge of the robot's system.
Na\"{\i}ve teachers may have greater assumptions on the robot's intelligence and take less care in executing demonstrations as compared to roboticists, who understand the effects of noisy demonstrations (\cite{suay2012practical}).
\cite{chen2003programing} and \cite{kaiser1995obtaining} recognised different sources for sub-optimality in demonstration such as the user demonstrating unnecessary or incorrect actions due the lack of knowledge about the task.
\cite{cakmak2014teaching} developed a PbD system and investigated the use of instructional materials in the form of tutorials and videos to support the learnability.

\textbf{Lack of comparison between algorithms.}
Despite there being many PbD algorithms proposed in research literature (\cite{argall2009survey,billing2010formalism}), there remains a lack in comparative user studies.
Difficulties in comparing across applications arise from the use of different robotic platforms and demonstration techniques, which lead to different representations of demonstration data.
\cite{suay2012practical} partly addresses this challenge by comparing three well-established algorithms and evaluating their performance in a common domain.


\subsection{On-going and Future work}
Future work will focus on exploring more challenging domains by including a wider range of predicates and probabilistic techniques to improve the condition inference.
In particular, the work for the remaining thesis duration will include the following:
\begin{itemize}
\item Complete user interface with options to modify planning domain, predicates, and define new problems.
\item Integrate automated planner using PDDL4Jrospy.
\item Use of multiple grippers to enable the grasping of different objects.
\item Complete integration of end-to-end system including action execution of newly taught actions.
\item User acceptance studies of the completed end-to-end system.
\item Validation and improvement of programming process and flow.
\end{itemize}

Possible extensions, which are out of scope of this thesis, but will be included if time permits, are as follows:
\begin{itemize}
	\item Use statistical methods for the generalisation of learned trajectories.
	\item Functionality to learn the colours and shapes of new object types.
	\item Use of multiple sensors to track the state of the world.
	\item Use of multiple cameras to enable the automatic recognition of all predicates.
	\item Functionality that compares newly created actions with already existing actions to avoid redundancies.
\end{itemize}