\section{Discussions}
\label{sec:discussions}
Both system and user evaluations demonstrated that iRoPro can be used to generalise primitive actions to a range of complex manipulation tasks and that it is easy to learn for users with or without programming experience.
In our system evaluation we could have programmed other actions, such as turning or pushing for packaging tasks.
As the purpose of our evaluation was to show the generalisability of primitive actions with the use of a task planner, we decided to stick to pick-and-place actions.
%Users particularly liked the PbD technique and found that
%`generating plans automatically' was the most useful feature 
In the following we discuss limitations and interesting extensions of our work:
\begin{enumerate}
	%\item No generalization from one item to another.
	\item Our object perception is limited as it does not detect objects that are too close together (\eg stacked objects).
	An improved perception system would allow the detection of initial states with stacked objects, automatically detecting goal states, or verifying action executions.
	%s mentioned in \sect{sec:implementation}, we partly the latter by using a mental model, where we first executed a stacking task to save the latest object positions, then reused the saved state for subsequent tasks.
	\item Due to the different grippers, we did not program actions that use both arms (\eg carrying a tray). A possible extension would be to include a better motion and task planning system in order to allow executing both arms simultaneously while avoiding self-collision.
	%As our robot was equipped with a suction and a claw gripper, the kinesthetic teaching of both arms could be difficult to program by a single user.
	\item We only included a minimal set of predicates (\sect{sec:highlevel}) that we deemed intuitive and useful for object manipulation tasks.
	It could be interesting to include and learn predicates to capture more complex domains such as object orientation \cite{li2016learning}.
	%the use a mobile robot that can move between workstations.
	%\item A possible extension would be to incorporate probabilistic techniques to learn predicates or pre-train the robot on simulated scenarios to improve the condition inference.
\end{enumerate}

\subsection{Challenges in PbD}\label{sssec:Challenges in PbD}
Despite ongoing research in PbD, there are several challenges that still need to be addressed:

\textbf{Suboptimality of demonstrations.}
A known PbD problem exists with regards to the type and quality of the demonstration which is dependent on the teacher's knowledge of the robot's system.
Na\"{\i}ve teachers may have greater assumptions on the robot's intelligence and take less care in executing demonstrations as compared to roboticists, who understand the effects of noisy demonstrations (\cite{suay2012practical}).
\cite{chen2003programing} and \cite{kaiser1995obtaining} recognised different sources for sub-optimality in demonstration such as the user demonstrating unnecessary or incorrect actions due the lack of knowledge about the task.
\cite{cakmak2014teaching} developed a PbD system and investigated the use of instructional materials in the form of tutorials and videos to support the learnability.

\textbf{Lack of comparison between algorithms.}
Despite there being many PbD algorithms proposed in research literature (\cite{argall2009survey,billing2010formalism}), there remains a lack in comparative user studies.
Difficulties in comparing across applications arise from the use of different robotic platforms and demonstration techniques, which lead to different representations of demonstration data.
\cite{suay2012practical} partly addresses this challenge by comparing three well-established algorithms and evaluating their performance in a common domain.


\section{On-going and Future work}
The main focus for the remainder of the thesis is a complete implementation of the end-to-end framework and a graphical interface to allow easy interaction with the robot. In particular, the work for the remaining thesis duration will include the following:
\begin{itemize}
\item Complete user interface with options to modify planning domain, predicates, and define new problems.
\item Integrate automated planner using PDDL4Jrospy.
\item Use of multiple grippers to enable the grasping of different objects.
\item Complete integration of end-to-end system including action execution of newly taught actions.
\item User acceptance studies of the completed end-to-end system.
\item Validation and improvement of programming process and flow.
\end{itemize}

Possible extensions, which are out of scope of this thesis, but will be included if time permits, are as follows:
\begin{itemize}
	\item Use statistical methods for the generalisation of learned trajectories.
	\item Functionality to learn the colours and shapes of new object types.
	\item Use of multiple sensors to track the state of the world.
	\item Use of multiple cameras to enable the automatic recognition of all predicates.
	\item Functionality that compares newly created actions with already existing actions to avoid redundancies.
\end{itemize}