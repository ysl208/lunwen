\section{Experiment 2: Acceptance of the Robot Programming Framework}
\label{sec:Exp2}

In this experiment, we address the following question:

\begin{enumerate}
  \item[\textbf{Q2}] Can users teach a robot action models for automated planning using the proposed framework iRoPro?
\end{enumerate}
\begin{sloppypar}
Users were presented a simulated implementation of iRoPro and had to teach action models by kinesthetically manipulating a Baxter robot (\fig{fig:Baxter}). 
Users were instructed to teach an atomic action by demonstration and assign preconditions and effects.
The goal was to assess the framework's usability and the user's difficulties encountered during the programming process.
At the end, participants were given a questionnaire related to their experience, their perceived understanding of the presented concepts and the usability of the framework.
In the following sections we briefly outline the experimental setup, measurements and results of the experiment.
%We then provide details on the partial implementation of the system used with the Wizard-of-Oz technique (\sect{ssec:WoZ}).
\end{sloppypar}

\subsection{Experimental Setup \& Participants}
We recruited 11 participants (7 male, 4 female), who were students and staff members at the Universit\'{e} Grenoble Alpes\footnote{None of the participants took part in the first experiment}.
6 participants reported programming experience with office productivity software (`beginner'), 2 had previously taken a programming course before (`advanced'), and 3 were pursuing studies in Computer Science (`expert').
%4 participants had previously heard of Automated Planning, but only 3 attended a related course.
The experiments were conducted using a Baxter robot, mounted with a partial implementation of the framework.
The implemented functionalities included:
\begin{itemize}
	\item `learn new action': record the kinesthetic action demonstration,
	\item `find a coloured object': apply the recorded action to an object of the specified colour,
	\item `execute an action sequence': execute a sequence of previously taught actions.
\end{itemize}

We used the Wizard-of-Oz technique to simulate the remaining functionalities (\eg `infer action preconditions and effects', `generate solution using a planner').
%Further details to the partial implementation are discussed in Section \ref{ssec:WoZ}.
Participants operated on a table with 2 positions D (for departure) and A (for arrival), 2 cubes (blue and red), that represented parts on an assembly line (\fig{fig:Baxter}). 
Each participant was allocated 1 hour, but the average duration was 29.5 minutes. 
The participants' behaviour was observed by the experimenter and the experiment was recorded on camera.
The experimental protocol, questionnaire and additional material used can be found in Appendix \ref{app:exp2}.

%The complete experimental protocol is shown in \fig{fig:Experimental protocol}. 
\subsection{Experimental Design \& Measurements}
The experiment scenario was set in a simulated assembly line, where objects of the same shape, but different colour arrived consecutively at the departure position D.
Users were told that objects were too heavy for human operators to move, hence needed to be handled by robots.
Due to the type of the objects, they should not be stacked either.
Users had to teach Baxter the action for moving an object from D to arrival position A, where another maintenance task would be performed later.
At the course of the experiment, users were faced with two different scenarios, where Baxter had to apply the learned move action. 
We evaluated the user's capability to refine action models and associate conditions when faced with different situations, and assessed the framework's overall usability.
The experiment consisted of the following phases:
\begin{itemize}
  %\item{Introduction: After a short introduction to the Baxter robot \cite{Baxter}, users were told that they needed to use a planning language (STRIPS) to explain Baxter the state of the world and the semantic meaning of the actions.}
  \item{\textbf{Training:} Users were shown how to manipulate Baxter's arm to pick and place an object, and given time to familiarise themselves with the kinesthetic manipulation. 
  	For this experiment we only used the robot's suction gripper to manipulate objects.}
  \item{\textbf{Experimental test:} Users were instructed to teach Baxter a move action of a red cube. 
  	Then, they were presented the action model, with preconditions and effects, that Baxter learned from the demonstration (\fig{fig:scenarios-exp2}a). 
  	In the following, users were faced with two different scenarios to refine the conditions of the action model, starting with the initial action model for a red cube.
  	At each step, users observed how Baxter executed the learned action in the new scenario. 
  	When Baxter failed to execute the action, users had to refine the conditions of the action model so that it was applicable to all cubes of any colour (\fig{fig:scenarios-exp2}b) and when the target position was occupied (\fig{fig:scenarios-exp2}c).
  }
  \item{\textbf{Planning:} Users were presented a new scenario, where Baxter was instructed to achieve a goal using the learned action model. 
  	The new goal was to switch the positions of two cubes on the table. 
  	Users were first asked if they believed Baxter was able to solve this task and were then shown how the taught action was reused with a task planner.
  	Finally, Baxter executed the action sequence to complete the task (\fig{fig:planning-permutation}).}
  \item{\textbf{Questionnaire:} At the end, users were given a questionnaire containing 18 questions related to their experience (\eg `I did not encounter any difficulties during the experiment'), their perceived understanding of the presented concepts (\eg `I can explain how Baxter represented the preconditions of a new action'), and the usability of the framework (\eg `No programming experience is required to teach Baxter a new task').
  Participants had to give a rating on a 4-point scale ranging from `Strongly agree', `Somewhat agree', `Somewhat disagree', and `Strongly disagree'.
  The complete questionnaire can be found in Appendix \ref{app:exp2}.}
   \item{ \textbf{Debriefing:} Throughout the experiment, users were asked about their expectations on Baxter's behaviour before applying the learned action model in a new scenario. 
   	Users were asked open-ended questions (\eg `What will Baxter do when applying the learned action model?'), so that their responses were unbiased. 
   	When they encountered failure scenarios (\eg when Baxter stacked two cubes), they were asked to reason about Baxter's behaviour and proposed modifications to the taught action model.} 
\end{itemize}

\begin{figure}[h]
	\centering
	\includegraphics[width=0.85\linewidth]{figures/scenarios-exp2}
	\caption{Continuous refinement of the move action model: (a) initial action model learned by demonstration, (b) action model for all cubes of any colour, (c) action model with an additional condition, if the target position is occupied and cubes can not be stacked.}
	\label{fig:scenarios-exp2}
\end{figure} 
\subsection{Results}
During the experiments we observed how users learned and used the presented programming process and planning concepts.
When asked for improvements of the initial action model no users pointed out missing conditions before being faced with the new scenario.
Even users who were `experts' and who have heard of automated planning before, did not propose a complete action model from the start. 
However, when faced with the relevant failure scenarios all of the users detected the missing conditions easily.
In the final phase, 8 (or 73\%) users who had no experience in automated planning did not expect Baxter to solve the permutation problem. 

Figure \ref{fig:eEvaluation} shows the user responses to the questionnaire.
All 11 users were satisfied with the programming process and Baxter's ability to learn and reproduce the demonstrated move action. 
All users stated that they encountered no difficulties during the experiment and believed that they had taught Baxter a new task. 
%All users understood the learned action model and managed to adopt the notions of preconditions and effects easily. 
The majority of the users agreed that they could explain how Baxter learned and represented the new action model.
9 (or 82\%) understood the notion of preconditions and agreed that no programming experience was required to teach Baxter using the proposed framework, while 2 (or 18\%) somewhat disagreed.


\begin{figure}[h]
	\centering
	\includegraphics[width=\linewidth]{figures/eEvaluation}
	\caption{Summary of questionnaire responses: Extract of 18 questions on the user's perceived usability and understanding of the programming process after the experiment.}
	\label{fig:eEvaluation}
\end{figure}